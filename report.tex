\documentclass[12pt, twoside]{article}
\usepackage[left=2.5cm,right=2cm,top=2cm,bottom=2cm]{geometry}

\usepackage{lastpage}
\usepackage{fancyhdr}
\pagestyle{fancy}
\renewcommand{\headrulewidth}{0.4pt}
\renewcommand{\footrulewidth}{0.4pt}
\renewcommand{\sectionmark}[1]{ \markright{#1}{} }
\lhead{Cyril NOVEL}\chead{}\rhead{\textit{ \nouppercase{\rightmark}}}
\lfoot{}\cfoot{\thepage\ of \pageref{LastPage}}\rfoot{}

\setlength{\headheight}{15pt}

\usepackage[french]{babel}
\usepackage[utf8]{inputenc}
\usepackage{amsmath}
\usepackage{graphicx}
%\usepackage{algpseudocode}
\usepackage{algorithm}
\usepackage{amsthm}
\usepackage{amssymb}
\usepackage{mathtools}
\usepackage{commath}
\usepackage{stmaryrd}
\usepackage{url}
\usepackage{caption}
\usepackage{subcaption}
\usepackage{hyperref}
\usepackage[noend]{algpseudocode}
\usepackage{listings}
\usepackage[usenames,dvipsnames,svgnames,table]{xcolor} % for setting colors
\usepackage{courier}

\let\oldsection\section
\def\section{\cleardoublepage\oldsection}

\renewcommand*\lstlistingname{Code}
% set the default code style
\lstset{
    frame=tb, % draw a frame at the top and bottom of the code block
    tabsize=2, % tab space width
    showstringspaces=false, % don't mark spaces in strings
    numbers=left, % display line numbers on the left
    commentstyle=\color{gray}, % comment color
    keywordstyle=\color{blue}, % keyword color
    stringstyle=\color{red}, % string color
    basicstyle=\footnotesize\ttfamily
}

\hypersetup{
  colorlinks=true,
  citecolor=black,
  urlcolor=Cerulean,
  linkcolor=PineGreen,
}

\title{Stage de fin d'études\\
\large{École polytechnique - Acute3D}}

\author{Cyril NOVEL}

\date{\today}

\begin{document}

\begin{titlepage}
\begin{center}
\includegraphics[width=0.20\textwidth]{LogoX.jpg}~\\[0.5cm]
\includegraphics[height=0.12\textwidth]{LogoA3D.jpg}~\\[1cm]

\textsc{\LARGE École polytechnique}\\[0.5cm]

\textsc{\Large Département d'informatique}\\[1.5cm]

% Title
\rule{\textwidth}{.4pt}
{ \huge \bfseries Sémantisation géométrique de modèles de villes : Détection de plans \\[0.4cm] }

\rule{\textwidth}{.4pt}\\[1.5cm]

% Author and supervisor
\begin{minipage}{0.4\textwidth}
\begin{flushleft} \large
\emph{Auteur:}\\
Cyril \textsc{Novel}
\end{flushleft}
\end{minipage}
\begin{minipage}{0.4\textwidth}
\begin{flushright} \large
\emph{Superviseur:} \\
Renaud \textsc{Keriven}
\end{flushright}
\end{minipage}

\vfill
{\large \today}
\end{center}
\end{titlepage}

\newpage
\begin{abstract}

\end{abstract}~\\[5cm]
%\addcontentsline{toc}{section}{Abstract}

\section*{Remerciements}
\addcontentsline{toc}{section}{Remerciements}
Je voudrais remercier mon superviseur Renaud Keriven, qui a su dirigé mes efforts, ainsi que Jean-Phillipe Pons. Je remercie aussi toute l'équipe d'Acute3D qui m'a réservé un accueil chaleureux et m'a permi de travailler dans un excellent environement.
\newpage

\tableofcontents
\newpage

\section*{Introduction}
\newpage

\section{Détection de plans}
\subsection{Problématique}
\subsection{RANSAC}
\subsection{Transformée de Hough}
\subsection{Croissance de région}
\subsection{Implémentation}

\section{Maillage}
\subsection{Problématique}
\subsection{Traitement du nuage de points}
\subsection{Classification des triangles}
\subsection{Implémentation}

\section{Simplification}
\subsection{Problématique}
\subsection{Simplification existante et limitations}
\subsection{Modifications}
\subsection{Implémentations}

\section{Résultats}
\subsection{Niveaux de simplifications}
\subsection{Temps d'éxécution}
\subsection{Problèmes existants}

\newpage
\section*{Conclusion}
\addcontentsline{toc}{section}{Conclusion}

\newpage
\bibliographystyle{plain}
\bibliography{biblio}
\addcontentsline{toc}{section}{References}
\end{document}