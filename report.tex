\documentclass[12pt, twoside]{article}
\usepackage[left=2.5cm,right=2cm,top=2cm,bottom=2cm]{geometry}

\usepackage{lastpage}
\usepackage{fancyhdr}
\pagestyle{fancy}
\renewcommand{\headrulewidth}{0.4pt}
\renewcommand{\footrulewidth}{0.4pt}
\renewcommand{\sectionmark}[1]{ \markright{#1}{} }
\lhead{Cyril NOVEL}\chead{Rapport de stage}\rhead{\textit{ \nouppercase{\rightmark}}}
\lfoot{}\cfoot{\thepage\ / \pageref{LastPage}}\rfoot{}

\setlength{\headheight}{15pt}

\usepackage[french]{babel}
\usepackage[utf8]{inputenc}
\usepackage{amsmath}
\usepackage{graphicx}
%\usepackage{algpseudocode}
\usepackage{algorithm}
\usepackage{amsthm}
\usepackage{amssymb}
\usepackage{mathtools}
\usepackage{commath}
\usepackage{stmaryrd}
\usepackage{url}
\usepackage{caption}
\usepackage{subcaption}
\usepackage{hyperref}
\usepackage[noend]{algpseudocode}
\usepackage{listings}
\usepackage[usenames,dvipsnames,svgnames,table]{xcolor} % for setting colors
\usepackage{courier}

\let\oldsection\section
\def\section{\cleardoublepage\oldsection}

\renewcommand*\lstlistingname{Code}
% set the default code style
\lstset{
    frame=tb, % draw a frame at the top and bottom of the code block
    tabsize=2, % tab space width
    showstringspaces=false, % don't mark spaces in strings
    numbers=left, % display line numbers on the left
    commentstyle=\color{gray}, % comment color
    keywordstyle=\color{blue}, % keyword color
    stringstyle=\color{red}, % string color
    basicstyle=\footnotesize\ttfamily
}

\hypersetup{
  colorlinks=true,
  citecolor=black,
  urlcolor=Cerulean,
  linkcolor=PineGreen,
}

\title{Stage de fin d'études\\
\large{École polytechnique - Acute3D}}

\author{Cyril NOVEL}

\date{\today}

\begin{document}

\begin{titlepage}
\begin{center}
\includegraphics[width=0.20\textwidth]{LogoX.jpg}~\\[0.5cm]
\includegraphics[height=0.12\textwidth]{LogoA3D.jpg}~\\[1cm]

\textsc{\LARGE École polytechnique}\\[0.5cm]

\textsc{\Large Département d'informatique}\\[1.5cm]

% Title
\rule{\textwidth}{.4pt}
{ \huge \bfseries Sémantisation géométrique de modèles de villes : Détection de plans \\[0.4cm] }

\rule{\textwidth}{.4pt}\\[1.5cm]

% Author and supervisor
\begin{minipage}{0.4\textwidth}
\begin{flushleft} \large
\emph{Auteur:}\\
Cyril \textsc{Novel}
\end{flushleft}
\end{minipage}
\begin{minipage}{0.4\textwidth}
\begin{flushright} \large
\emph{Superviseur:} \\
Renaud \textsc{Keriven}
\end{flushright}
\end{minipage}

\vfill
{\large \today}
\end{center}
\end{titlepage}

\newpage
\begin{abstract}
Dans ce rapport,
\end{abstract}~\\[5cm]
\addcontentsline{toc}{section}{Résumé}

\section*{Remerciements}
\addcontentsline{toc}{section}{Remerciements}
Je voudrais remercier mon superviseur Renaud Keriven, qui a su dirigé mes efforts, ainsi que Jean-Phillipe Pons. Je remercie aussi toute l'équipe d'Acute3D qui m'a réservé un accueil chaleureux et m'a permi de travailler dans un excellent environement.
\newpage

\tableofcontents
\newpage

\section*{Introduction}
\addcontentsline{toc}{section}{Introduction}
La création de modèles tridimensionnels de villes à partir de données photographiques et/ou de relevés lasers de manière automatique est en plein essor. De nombreux acteurs, tels Google, Apple ou Microsoft, proposent de visualiser des villes et des bâtiments en trois dimensions avec leur logiciels de cartographie respectifs. Les problèmatiques soulevées sont nombreuses et le problème initial consistant à obtenir un modèle précis et détaillé a été longuement étudié et de nombreuses techniques permettent d'obtenir des résultats satisfaisant.

De nos jours, la diffusion de ces modèles se fait via Internet. Il est alors crucial de réduire le poids du modèle créé, tout en maintenant un niveau de détail suffisant. Il est nécessaire de simplifier intelligemment le modèle, en prenant en compte la géométrie et la topologie de la reconstruction. On parle alors de sémantisation du modèle. Avant de repérer des structures complexes, comme les types de bâtiments ou les détails architecturaux, la première étape consiste à identifier des primitives géométriques -- plans, cylindres, quadriques -- pour simplifier le modèle. Si des travaux existent à ce sujet, ils ne prennent pas en compte les spécificités des modèles 3D de villes, comme le manque de précision et le bruit ainsi que la variation de l'échantillonage au sein du modèle.

Le but du stage est de créer et développer un algorithme permettant d'identifier des plans au sein de modèles 3D variés -- villes, bâtiments -- et de simplifier le modèle en accordance avec les plans détectés.
\newpage

\section{Détection de plans}
\subsection{Problématique}
Détecter des plans est la première étape de notre algorithme global. Avant tout, il a fallu choisir l'objet qui allait être traiter. Le logiciel \textit{Smart3DCapture} génèrent plusieurs objets tout au long de la chaîne de traitement. Un nuage de point bruité est généré, suivi par un maillage 3D grossier et ensuite d'un maillage 3D précis. Il aurait été possible d'extraire des plans du nuage bruité. Cependant des détails -- fenêtres, cheminées -- auraient être confondus avec un plan. Il a donc été décidé d'extraire des plans à partir du maillage 3D précis.

Le maillage possède aussi un avantage certain sur le nuage de points. De nombreuses informations sont plus simples et plus rapide à calculer grâce à la connectivité du maillage. L'estimation de l'échantillonage devient triviale, de même que pour l'estimation de la normale.

Dans la suite, nous détaillons les techniques les plus courantes pour extraire des plans d'un nuage de points ou d'un maillage, les deux pouvant s'appliquer à notre maillage.

\subsection{RANSAC}
Le \textit{RANSAC} -- \textit{RANdom SAmple Consensus} -- est un algorithme itératif non déterministe. Soit un nuage de points $C$. À chaque itération de l’algorithme, on choisit un nombre $p$ de points au sein du nuage de points de manière aléatoire. On note $S$ cet ensemble. On calcule alors le meilleur plan pour cet ensemble de points, c’est à dire le plan qui minimise la somme des distances au carré de points à ce plan -- cf annexe. Une fois le plan trouvé, on parcourt les points de $C\setminus S$. Si le point correspond au plan avec une certaine erreur $d$, on l’ajoute à l’ensemble $S$. Une fois l’ensemble des points de $C\setminus S$ parcouru, on recalcule le meilleur plan pour l’ensemble $S$. Si le plan est meilleur que celui de l’itération précédente -- la somme des distances au carré est plus faible, alors on le considère comme le meilleur plan courant. Sinon le plan de l’itération précédente reste le meilleur plan courant. On procède ainsi pendant $k$ itérations, $k$ étant fixé à l’avance. Une fois le meilleur plan trouvé, $C$ = $C\setminus S$ et on effectue un nouveau \textit{RANSAC} sur $C$ pour trouver le prochain plan. Il est possible de s’arrêter avant la $k$ème itération si une itération est suffisamment bonne, c’est à dire que l’erreur est suffisamment faible.

\textit{RANSAC} a l’avantage de posséder une excellente robustesse aux outliers XXX BIBLIO[RL94]. Cependant, l’algorithme est lent. Pour les $k$ itérations, on ne trouve qu’un seul plan. Il faut donc répéter la méthode $n$ fois pour trouver $n$ plans. Le nombre de plans est totalement dépendant de la scène considérée, il est difficile d’estimer ce paramètre correctement. Une solution peut être de fixer une erreur minimale maximale. Si au $i^\text{ème}$ plan, à la $k^\text{ème}$ itération, l’erreur minimale est supérieure à un certain seuil, alors on arrête la recherche. Une autre amélioration consiste à sélectionner les points dans un certain voisinage, afin d’augmenter la probabilité de détecter un plan.

Un autre problème réside dans la taille variable des plans. Dans un modèle de ville, les rues sont planes tout comme certains toits. Les rues contiennent plus de 50000 points alors que un pan de toit va contenir moins de 700 points. Le choix de la variable $p$ est difficile. Il faudrait l'initialiser à moins de 700 points pour espérer obtenir le pan de toit. Sachant que le nuage peut contenir plus de 2 millions de points, le temps d'exéction du \textit{RANSAC} devient extrêmement long.

Afin de raccourcir le temps d'exécution, il est possible de segmenter le nuage de points en une grille 3D, comme proposé par XXX BIBLIO[YF10]. Pour chaque bloc un algorithme de RANSAC est lancé. Facilement parallélisable, cette technique a aussi l'avantage d'être moins coûteuse -- moins de points à tester. Le choix de la variable $p$ est moins compliqué puisque la taille des plans est limité par la taille des cases. Cependant, la segmentation crée d'autres problèmes. Si un plan présent dans une case $C_1$ déborde légèrement dans une case $C_2$, il ne sera pas détecté dans cette dernière. La détection des plans n'est donc pas optimale. D'autre part, chaque plan est segmenté sur chaque case. Si un plan traverse $n$ cases, il y aura $n$ équation différentes pour ce plan. Une étape supplémentaire de fusion doit donc être considérée pour obtenir un résultat acceptable.

À partir de ces éléments, \textit{RANSAC} n'est pas adapté pour notre problématique.

\subsection{Transformée de Hough}

\subsection{Croissance de région}

\subsection{Implémentation}

\section{Maillage}
\subsection{Problématique}
\subsection{Traitement du nuage de points}
\subsection{Classification des triangles}
\subsection{Implémentation}

\section{Simplification}
\subsection{Problématique}
\subsection{Simplification existante et limitations}
\subsection{Modifications}
\subsection{Implémentation}

\section{Résultats}
\subsection{Niveaux de simplifications}
\subsection{Temps d'éxécution}
\subsection{Gain de place}
\subsection{Problèmes existants}

\newpage
\section*{Conclusion}
\addcontentsline{toc}{section}{Conclusion}

\newpage
\bibliographystyle{plain}
\bibliography{biblio}
\addcontentsline{toc}{section}{References}
\end{document}