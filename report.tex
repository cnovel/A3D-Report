\documentclass[12pt, twoside]{article}
\usepackage[left=2.5cm,right=2cm,top=2cm,bottom=2cm]{geometry}

\usepackage{lastpage}
\usepackage{fancyhdr}
\pagestyle{fancy}
\renewcommand{\headrulewidth}{0.4pt}
\renewcommand{\footrulewidth}{0.4pt}
\renewcommand{\sectionmark}[1]{ \markright{#1}{} }
\lhead{Cyril NOVEL}\chead{}\rhead{\textit{ \nouppercase{\rightmark}}}
\lfoot{}\cfoot{\thepage\ of \pageref{LastPage}}\rfoot{}

\setlength{\headheight}{15pt}

\usepackage[french]{babel}
\usepackage[utf8]{inputenc}
\usepackage{amsmath}
\usepackage{graphicx}
%\usepackage{algpseudocode}
\usepackage{algorithm}
\usepackage{amsthm}
\usepackage{amssymb}
\usepackage{mathtools}
\usepackage{commath}
\usepackage{stmaryrd}
\usepackage{url}
\usepackage{caption}
\usepackage{subcaption}
\usepackage{hyperref}
\usepackage[noend]{algpseudocode}
\usepackage{listings}
\usepackage[usenames,dvipsnames,svgnames,table]{xcolor} % for setting colors
\usepackage{courier}

\let\oldsection\section
\def\section{\cleardoublepage\oldsection}

\renewcommand*\lstlistingname{Code}
% set the default code style
\lstset{
    frame=tb, % draw a frame at the top and bottom of the code block
    tabsize=2, % tab space width
    showstringspaces=false, % don't mark spaces in strings
    numbers=left, % display line numbers on the left
    commentstyle=\color{gray}, % comment color
    keywordstyle=\color{blue}, % keyword color
    stringstyle=\color{red}, % string color
    basicstyle=\footnotesize\ttfamily
}

\hypersetup{
  colorlinks=true,
  citecolor=black,
  urlcolor=Cerulean,
  linkcolor=PineGreen,
}

\title{Stage de fin d'études\\
\large{École polytechnique - Acute3D}}

\author{Cyril NOVEL}

\date{\today}

\begin{document}

\begin{titlepage}
\begin{center}
\includegraphics[width=0.20\textwidth]{LogoX.jpg}~\\[0.5cm]
\includegraphics[height=0.12\textwidth]{LogoA3D.jpg}~\\[1cm]

\textsc{\LARGE École polytechnique}\\[0.5cm]

\textsc{\Large Département d'informatique}\\[1.5cm]

% Title
\rule{\textwidth}{.4pt}
{ \huge \bfseries Sémantisation géométrique de modèles de villes : Détection de plans \\[0.4cm] }

\rule{\textwidth}{.4pt}\\[1.5cm]

% Author and supervisor
\begin{minipage}{0.4\textwidth}
\begin{flushleft} \large
\emph{Auteur:}\\
Cyril \textsc{Novel}
\end{flushleft}
\end{minipage}
\begin{minipage}{0.4\textwidth}
\begin{flushright} \large
\emph{Superviseur:} \\
Renaud \textsc{Keriven}
\end{flushright}
\end{minipage}

\vfill
{\large \today}
\end{center}
\end{titlepage}

\newpage
\begin{abstract}
The Oculus Rift is a device capable of displaying 3D information on the whole field of view of the user. The Kinect is a RGBD camera, capable of computing a depthmap of the scene it is looking at. In this project, we propose to combine the Oculus Rift and the Kinect to create a new kind of augmented reality device. With the algorithm KinectFusion, we are capable of computing a 3D representation of the surrounding scene and display it into the Rift for the user. To reinforce the immersion, we introduce live color projection on the scene. The Rift and the Kinect are correctly calibrated to provide the best user experience.

We also present an augmented reality application called \textit{magical pen} where the user can freely draw on a surface.
\end{abstract}~\\[5cm]
%\addcontentsline{toc}{section}{Abstract}

\section*{Acknowledgements}
\addcontentsline{toc}{section}{Acknowledgements}
I would like to thank Professor Andrew Davison for his help and indications throughout this project. Thank you to Steven Lovegrove, who helped me on the crucial point of Lens distortion for Rift rendering.
\newpage

\tableofcontents

\newpage

\section*{Introduction}

\newpage

\section{Hardware review}
\subsection{Virtual reality headsets}

\newpage
\section*{Conclusion}
\addcontentsline{toc}{section}{Conclusion}
The combination of Kinect and Oculus Rift we proposed is working almost perfectly. The device, even if is is uncomfortable, is usable and provides a strong proof of concept for the future of augmented reality devices. The user is capable of moving in the scene without stumbling around: the live color projection allows a better immersion into the 3D scene and prevents the awkward feeling of being somewhere else. The range of the reconstruction is acceptable -- a $5m^3$ box -- with a good resolution -- details such as space between keyboard keys are noticeable.

The augmented reality application \textit{Magical Pen} is simple but demonstrates the feasibility of AR applications with the device. It doesn't impact the reconstruction rate or the framerate of the rendering, making it transparent to the user. The large field of view covered by the Virtual Reality Headset suggests that Google Glass-like applications are easily transposable to our device. The coverage of the whole field of view will allow more natural integration of interaction between the user and the device.

The interest in Virtual Reality and Augmented Reality is still growing, with companies like Facebook, Apple or Google being highly interested in these fields. The Project Tango devices are almost consummer-ready smartphones and tablets capable of capturing a scene and interacting with it. It is likely to see in the next decade an improved version of our prototype for industrial or professional purpose.


\newpage
\bibliographystyle{plain}
\bibliography{biblio}
\addcontentsline{toc}{section}{References}
\end{document}